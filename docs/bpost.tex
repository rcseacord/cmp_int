\documentclass[11pt,letterpaper]{article}
%\usepackage{ncc_whitepaper}
\usepackage[english]{babel}
\documentclass{article}
\usepackage{xcolor}
\usepackage{listings}

\definecolor{mGreen}{rgb}{0,0.6,0}
\definecolor{mGray}{rgb}{0.5,0.5,0.5}
\definecolor{mPurple}{rgb}{0.58,0,0.82}
\definecolor{backgroundColour}{rgb}{0.95,0.95,0.92}

\lstdefinestyle{CStyle}{
    backgroundcolor=\color{backgroundColour},   
    commentstyle=\color{mGreen},
    keywordstyle=\color{magenta},
    numberstyle=\tiny\color{mGray},
    stringstyle=\color{mPurple},
    basicstyle=\footnotesize,
    breakatwhitespace=false,         
    breaklines=true,                 
    captionpos=b,                    
    keepspaces=true,                 
    numbers=left,                    
    numbersep=5pt,                  
    showspaces=false,                
    showstringspaces=false,
    showtabs=false,                  
    tabsize=2,
    language=C
}

\lstdefinestyle{CppStyle}{
    backgroundcolor=\color{backgroundColour},   
    commentstyle=\color{mGreen},
    keywordstyle=\color{magenta},
    numberstyle=\tiny\color{mGray},
    stringstyle=\color{mPurple},
    basicstyle=\footnotesize,
    breakatwhitespace=false,         
    breaklines=true,                 
    captionpos=b,                    
    keepspaces=true,                 
    numbers=left,                    
    numbersep=5pt,                  
    showspaces=false,                
    showstringspaces=false,
    showtabs=false,                  
    tabsize=2,
    language=C++
}



\newcommand{\papertitle}{Tool Release: Safe Integer Comparison in C}
\newcommand{\paperabstract}{\blindtext}
\newcommand{\docversion}{1.0}


\title{Tool Release: Safe Integer Comparison in C}
\author{Anton Gerasimov\\
Robert Seacord\\
Aaron Ballman}


% \newcommand{\draft} % Comment this out when distributing
\draftcheck

\makeindex

\begin{document}
%%% Header, ToC and title
\maketitle
\newpage
\thispagestyle{fancy}
\clearpage
\section*{Table of Contents}
\makeatletter
\renewcommand\tableofcontents{\@starttoc{toc}} % Remove heading from ToC
\makeatother
\bgroup%
%%%\addfontfeatures{BoldFeatures={Color=nccred}}
\hypersetup%
\tableofcontents
\egroup%
\thispagestyle{nccsection}
\pagebreak
%%%


\section{Introduction}

In this paper, we discuss the challenges that arise from integer comparison in C (Section~
\ref{sec:overview}) and describe the library that we are releasing (Section~\ref{sec:cimplementation}), and compare it with the C++ implementation (Section~\ref{sec:cppimplementation}). 
We also give suggestions for use (Section~\ref{sec:suggestions}) and directions of future work (Section~\ref{sec:futurework}).

\section{Overview– Integer Security in C}
\label{sec:overview}

\subsection{Background }

Signed / unsigned comparison is a major source of defects and software vulnerabilities in C and C++ programming [Seacord 2005, Stroustrup 2019].

@book{10.5555/2530475,
author = {Seacord, Robert C.},
title = {Secure Coding in C and C++},
year = {2013},
isbn = {0321822137},
publisher = {Addison-Wesley Professional},
edition = {2nd},
}

Bjarne Stroustrup.  "Don’t add to the signed/unsigned mess". 2019-02-14 https://www.open-std.org/jtc1/sc22/wg21/docs/papers/2019/p1491r0.pdf

For example, the following expression:

\begin{lstlisting}[style=CStyle]
 -1 > 0u 
\end{lstlisting}

evaluates to \texttt{true} which can be surprising to inexperienced programmers.
This particular comparison is between two integer literals, so the first step in this process is determining the type of these literals.
Even this process is complicated as the type depends on the implementation-defined ranges of integer types, the values, and any suffixes used by the integer literal.
The left-hand side of this comparison has the value \texttt{-1}.
This is actually the integer literal one preceeded by the unary negation operator.
The integer literal 1 in this example has the type \texttt{int}.
The right-hand side of this expression consists solely of the integer literal \texttt{0u}.
The \texttt{u} in this example is an integer literal suffix meaning unsigned.
The integer literal \texttt{0u}  has the type \texttt{unsigned int}.
Before the comparison can be evaluated expression, C must balance the two operands to a common type.
The process for doing so is called the \emph{usual arithmetic conversions}.
To determine the common type, each type is assigned a conversion rank according to the \emph{integer conversion rank} rules.
The rank of any unsigned integer type is equal to the rank of the corresponding signed integer type.
If the operand that has unsigned integer type has rank \texttt{>=} the rank of the operand with signed integer type, the operand with signed integer type is converted to the type of the operand with unsigned integer type.
In this particular example, this means that \texttt{-1} is converted to the type \texttt{unsigned int}, and the comparison is performed between two integer values of this type.
Because unsigned types cannot represent negative values, when \texttt{-1} is converted to \texttt{unsigned int} the bit pattern represents the value \texttt {UINT\_MAX} the maximum value that can be represented as \texttt{unsigned int}.
At this point, it should be clear why \texttt{-1 > 0u} evaluates to \texttt{true}, as \texttt {UINT\_MAX} is greater than zero.

The general \emph{integer conversion rank} rules and \emph{usual arithmetic conversions} are more complicated of course.
This example does not even involve \emph{integer promotions}, which are only applied to an object or expression with an integer type whose integer conversion rank is less than or equal to the rank of \texttt{int} and \texttt{unsigned int}.  

Any comparison operator (==,=!,\textless=,\textless,\textgreater=,\textgreater) can provoke an implicit type conversion when the operands are of different types. 
This can occur in a number of ways. 
The following example is similiar to the previous one, but uses objects with defined, but different integer types:

\begin{lstlisting}[style=CStyle]
signed int x=-1;
unsigned int y=1;
printf(“beware the output of x < y is actually %d\n”, x < y);
\end{lstlisting}

Again, the rvalue on the left-hand side of the comparison operator is converted from signed int to unsigned int as a result of the usual arithmetic conversion.
The result is that the expression \texttt{x < y} evaluates to false.

We consider now the following:

\begin{lstlisting}[style=CStyle]
signed char user_input_len;
char user_input_buffer[]="msg";
user_input_len=get_unvalidated_user_input(...);

if(user_input_len==sizeof(user_input_buffer)){
	printf("continue unintended program execution");
	}	
 	
\end{lstlisting}

In this example,  \texttt{user\_input\_len} is promoted to \texttt{unsigned int} as explained above. However, if we now suppose that the user has populated \texttt{user\_input\_len} as -252, then the equality check in the if statement will in fact be verified, and thus this validation check has been bypassed because of the unintended behavior of the implicit type conversion. 

\subsection{Existing palliative measures}

\subsubsection{Compiler flags and warnings}
The most immediate way to address implicit conversions arising because of comparison is to use compiler flags. In GCC, the relevant flags are -Wsign-compare, which checks for comparisons between signed and unsigned types, as well as -Wsign-conversion, which warns that an implicit conversion may have changed the sign of a conversion. Both of these checks are included in the -Wextra flag. Given the inclination of developers to ignore warnings, more recent versions of GCC allow the -Werror=specific-warning flag to replace an error with a warning.

(https://gcc.gnu.org/onlinedocs/gcc/Warning-Options.html)

In Clang, the main flag is -Wsign-conversion, which also controls the more granular

-Wimplicit-int-conversion, -Wint-conversion, and -Wsign-conversion.

(https://clang.llvm.org/docs/DiagnosticsReference.html#wimplicit-int-conversion)

In MSVC, like with GCC and Clang, the default setting does not warn about implicit conversions. There is however the C4018 level 3 compiler warning , which notifies the developer that an operator had forced an implicit conversion from signed to unsigned, with the same trigger for level 4 warning C4388. Likewise, level 4 warning C4389 notifies that an "==" or "!=" operation involved signed and unsigned variables. However, these warnings must be used with care as it has been detected that error C4018 is not triggered when the operands are of different size, and Microsoft has closed the ticket without resolving the issue, as it was deemed of insufficient importance. 

(https://learn.microsoft.com/en-us/cpp/error-messages/compiler-warnings/compiler-warning-level-3-c4018?view=msvc-170

https://learn.microsoft.com/en-us/cpp/error-messages/compiler-warnings/c4388?view=msvc-170

https://learn.microsoft.com/en-us/cpp/error-messages/compiler-warnings/compiler-warning-level-4-c4389?view=msvc-170

https://developercommunity.visualstudio.com/t/warning-c4018-does-not-work-when-the-sizes-of-type/335285)

\subsubsection{Solutions in code}
C does not have the \texttt{explicit} keyword in C++ that bans implicit conversions. There are not many options to prevent this problem in code. The first would be to type cast within the comparison. This far from elegant option does indeed have the merit in  forcing the developer to ask themselves what is the type of the variables inside the comparison. However, this only shifts the problem, as then it creates a risk of an error relating to type casting. Furthermore, this would be a somewhat unnatural demand to place upon developers in isolation. 

The more natural option is to use specific industry standards like Cert C or MISRA.  

The Motor Industry Software Reliability Association (MISRA) has developed a set of guidelines for C development in embedded and safety critical applications called MISRA C. These guidelines include an "Essential Type Model", which MISRA describes in the following way:

"The essential type of an object or expression is defined by its essential type category and size. The essential type category of an expression reflects its underlying behavior and may be:

Essentially Boolean;
Essentially character;
Essentially enum;
Essentially signed;
Essentially unsigned;
Essentially floating;"

With this model, of interest to the present discussion are the following two rules:

Rule 10.1: "Operands shall not be of an inappropriate essential type"
Rule 10.4: "Both operands of an operator in which the usual arithmetic conversions are performed"

It is not the objective of this article to discuss MISRA at length, but with regards to safe comparison the following observations can be made. For the most part implicit conversions arise as a consequence of operators being of different signedness, so the above restriction goes a long way in making integer comparison safer. 

However, MISRA rules are not applied in isolation. There is a large number of restrictions on how the language can be used, and compliance requires a high level of commitment. Although there is no formal MISRA certification, adherence to the rules would require adoption from the outset of the project, and requires appropriately skilled staff, necessary operational procedures, and extensive documentation. For these reasons MISRA compliance typically makes sense in safety critical applications, for which it was designed. 

\section{C++ Solution}
\label{sec:cppimplementation}

The C++20 standard has implemented safe integer comparison functions in the Utilities library.

(https://en.cppreference.com/w/cpp/utility/intcmp)

(https://www.open-std.org/jtc1/sc22/wg21/docs/papers/2020/p0586r2.html)

In the GNU C++ Standard Library, the utility functions are in the "Utility" Header, and in turn relies heavily on the make\_unsigned function. The latter is defined in the "Typetraits", and makes use of metaprogramming templates. 

Template metaprogramming is not a feature of the C language. Furthermore, metaprogramming is not suitable for kernel/driver, embedded, and safety-critical applications, which favor the imposition of rigid constraints on development practices. In these programming contexts, less is more, as increased functionality presents a security and safety risk. 

\blindtext
\blindlist{itemize}

\section{C Solution}
\label{sec:cimplementation}
\subsection{Our definition}
It was the objective of the authors to provide developers with specific utility functions to prevent integer comparison vulnerabilities, without having to enforce wider and more cumbersome standards. 

We have provided a C definition of integer comparison functions, employing the same arithmetic logic as the definitions of the C++ Standard, but using builtins from the C11 standard. Specifically, the definitions heavily rely on the \_generic selection newly introduced in C11, which selects an expression at compile time based on the type of the argument. This keyword provides a capability similar to type overloading in C++, and therefore allows a natural transposition of the utility functions.

To take cmp\_less as an example, the GCC implementation of the C++ standard library defines: 

\begin{lstlisting}[style=CppStyle]
  template<typename _Tp, typename _Up>
    constexpr bool
    cmp_less(_Tp __t, _Up __u) noexcept
    {
      static_assert(__is_standard_integer<_Tp>::value);
      static_assert(__is_standard_integer<_Up>::value);

      if constexpr (is_signed_v<_Tp> == is_signed_v<_Up>)
	return __t < __u;
      else if constexpr (is_signed_v<_Tp>)
	return __t < 0 || make_unsigned_t<_Tp>(__t) < __u;
      else
	return __u >= 0 && __t < make_unsigned_t<_Up>(__u);
    }
\end{lstlisting}

Which in our C implementation (the non-portable definition) becomes: 

\begin{lstlisting}[style=CStyle]
#define cmp_less(lhs, rhs)                                                     
  ({                                                                           
    __typeof__(lhs) lhs_val = lhs;                                             
    __typeof__(rhs) rhs_val = rhs;                                             
    check(lhs_val, "lhs");                                                     
    check(rhs_val, "rhs");                                                     
    int res = 0;                                                               
    if (is_signed(lhs_val) == is_signed(rhs_val))                              
      res = (lhs_val) < (rhs_val);                                             
    else if (is_signed(lhs_val))                                               
      res = (lhs_val) < 0 ? 1 : make_unsigned(lhs_val) < (rhs_val);            
    else                                                                       
      res = (rhs_val) < 0 ? 0 : (lhs_val) < make_unsigned(rhs_val);            
    res;                                                                       
  })
\end{lstlisting}

With the make\_unsigned operator is defined as:

\begin{lstlisting}[style=CStyle]
#define make_unsigned(x) (_Generic((x), 
                         default: make_unsigned_fixed_width(x), 
                         signed char: ((unsigned char)x), 
                         signed short: ((unsigned short)x), 
                         signed int: ((unsigned int)x), 
                         signed long: ((unsigned long)x), 
                         signed long long: (unsigned long long)x))
                                          
#define make_unsigned_fixed_width(x) (_Generic((x), 
                         default: (x), 
                         int8_t: ((uint8_t)x), 
                         int16_t: ((uint16_t)x), 
                         int32_t: ((uint32_t)x), 
                         int64_t: ((uint64_t)x)))
\end{lstlisting}


Unlike the builtin C comparison operators, negative signed integers always compare "less than" (and "not equal to") unsigned integers: the comparison is safe against lossy integer conversion. It is a compile-time error if either the left hand side or right hand side is not a signed or unsigned integer type (including standard integer type and extended integer type). It is also a compile-time error if the either left hand side or right hand side is the char and bool type. While technically integer types, the char type should only be used to represent character values where the signedness is unimportant and the 'bool' type is always unsigned.

\subsection{Portable/no}
The header provides 2 types of functions, portable and not. 

The CMP\_INT\_WANT\_PORTABLE macro is defined to 1 for the most portable implementation of these facilities. This macro will disable the use of GNU statement expressions and other extensions that protect against unintended duplicate side effects during macro expansion. For this reason, the macro defaults to requesting the nonportable interfaces that are more hygenic.

These interfaces require compiling with support for C11 functionality, even when requesting maximal portability.

\subsection{Performance differences}

\section{Suggestions For Use}
\label{sec:suggestions}

In the below section, we have a simple program that returns an incorrect comparison within an if statement. As discussed earlier in this article, we have a comparison between an unsigned and unsigned integer which yields an unexpected verification. 

\begin{lstlisting}[style=CStyle]
int main(void)
{
       unsigned int len=1;

        if (len<-1){
        printf("1 is less than -1, congratulations\n");
        }

        return 0;
        
}
\end{lstlisting}


In the below, we replace the comparison with the cmp\_less function, which remediates the problem and returns the expected result. 

\begin{lstlisting}[style=CStyle]
int main(void)
{
        unsigned int len=1;

        if (cmp_less(len,-1)){
        printf("this should not print anything\n");
        }

        if (cmp_greater(len,-1)){
        printf("this is the correct answer\n");
        }

        if (cmp_equal(len,-1)){
        printf("this should not print anything\n");
        }

        return 0;

}
\end{lstlisting}



To take a more involved real world example, we take the UnixODBC driver vulnerability 

https://nvd.nist.gov/vuln/detail/CVE-2011-1145






\section{Future Work}
\label{sec:futurework}
\printindex
\end{document}
